\documentclass{article}
\usepackage[english]{babel}
\usepackage[utf8]{inputenc}
\usepackage[T1]{fontenc}
\usepackage[xindy,original]{imakeidx}
\usepackage[acronym, toc,numberedsection=nolabel]{glossaries}
\usepackage[backend=biber,style=authoryear]{biblatex}
\makeindex[intoc]
\makeglossaries
\usepackage[]{url}
%\newacronym{
\begin{document}
	\title{TeX4ebook showcase}
	\author{Michal Hoftich}
	\maketitle
	\tableofcontents
	\section{Intro}

	This is sample document for \verb|tex4ebook|%
	\footnote{\url{https://github.com/michal-h21/tex4ebook}}, 
	showing advanced \LaTeX features like different scripts and languages, 
	indexing, glossaries, biblatex bibliography and tikz figures.
	
	TeX4ebook is based on TeX4ht, system for converting \LaTeX documents to 
	various output formats. We use it to convert \LaTeX to xhtml or html5, 
	and then pack it using lua scripts to epub, epub3 or mobi formats.

	For configuring html and css output of various \LaTeX commands, 
	we can use custom config file. You can see config file in source code
	repository of this document%
	\footnote{\url{https://github.com/michal-h21/epub3sample/blob/master/latex-showcase/epub3.cfg}},
	
	
	\index{index entry!sample}
	\section{Second}
	Hello world
	\printglossaries
	\printbibliography
\AtBeginEnvironment{theindex}{Ahoj, rejstříku \addcontentsline{toc}{section}{\indexname}}
	\printindex
\end{document} 
